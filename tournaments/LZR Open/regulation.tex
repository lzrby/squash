\documentclass[12pt, a4paper]{extarticle}

\usepackage{geometry}
\usepackage{titlesec}
\usepackage[utf8]{inputenc}
\usepackage[english, russian]{babel}
\usepackage[normalem]{ulem}
\usepackage{soul}
\usepackage{verbatim}
\usepackage{tabularx}
\usepackage{tabulary}
\usepackage{hyperref}
\usepackage{setspace}
\usepackage{amssymb,amsfonts,amsmath,cite,enumerate,float,indentfirst}
\usepackage{mathrsfs}
\usepackage{comment}
\usepackage{verbatim}
\usepackage{mathtools}
\usepackage{caption}
\usepackage{wrapfig}
\usepackage{mathtext} %cyrillic text in math mode, not recommended
\usepackage{amsthm}
\usepackage{graphicx}
\usepackage{setspace}
\usepackage{enumitem}
\usepackage[table]{xcolor}

\geometry{left=20mm, right=10mm, top=20mm, bottom=20mm}

\setlist[itemize]{topsep=0pt,partopsep=1ex,parsep=1ex}

\titleformat{\section}[block]{\Large\bfseries\filcenter}{}{1em}{}
\titleformat{\subsection}[block]{\large\bfseries\filcenter}{}{1em}{}

\hypersetup{				
    unicode=true,           
    pdfkeywords={keyword1} {key2} {key3},
    colorlinks=true, 
    urlcolor=blue,
    linkcolor=black
}

\setlength{\parindent}{1.3em}
\setlength{\parskip}{10pt}
\pagenumbering{gobble}

% figure captions settings
\captionsetup[figure]{labelfont=bf, justification=centering}
\renewcommand\thefigure{\thesection.\arabic{figure}}  
\makeatletter
\renewcommand{\fnum@figure}{Малюнак \thefigure}
\makeatother

% table captions settings
\captionsetup[table]{justification=centering, singlelinecheck=false, font=large}
\makeatletter
\renewcommand{\fnum@table}{}
\makeatother
\renewcommand\thetable{\thesection.\arabic{table}}  
\makeatletter
\renewcommand{\fnum@table}{Табліца \thetable}
\makeatother

\newcommand{\formQA}[2]{%
    \noindent \textbf{Q:} #1 \\
    \textbf{A:} #2
}

\begin{document}
    \section{Рэгламент}
    
    \subsection{Групавы этап (01 верасня "--- 30 лістапада)}

    Кожны з кожным гуляе матч у фармаце best of 5 (да 3 перамог). Вынікі вызначаюцца па колькасці перамог. 
    
    Пры роўнай колькасці перамог у першую чаргу будуць глядзецца:
    \begin{enumerate}
        \item Колькасць згуляных матчаў.
        \item Вынік асабістай гульні.
        \item Розніца паміж колькасцю выйграных і прайграных партый унутры 3+ чалавек, для якіх трэба вызначыць месца.
        \item Колькасць выйграных партый унутры 3+ чалавек, для якіх трэба вызначыць месца.
        \item Агульная розніца паміж колькасцю выйграных і прайграных партый.
        \item Агульная колькасць выйграных партый.
        \item Вынік гульні з гульцом, які дакладна заняў $i$ месца, $i = \overline{1, n}$, дзе n "--- колькасць удзельнікаў.
    \end{enumerate}
    
    Калі матч не быў згуляным і нельга усталяваць, хто цалкам у гэтым вінаваты (адхіляўся ад гульні), у табліцу заносіцца лік 0:0, але матч не лічыцца згуляным, у іншым выпадку вінаваты атрымлівае тэхнічную паразу 0:3.
    
    \subsection{Плэй-оф (01-31 снежня)}
    
    У плэй-оф выходзяць 6 лепшых гульцоў і гуляюць па наступнай схеме:
    
    \begin{itemize}
        \item Чвэрцьфінал: матч да 3 перамог. Першае і другое месца групы пачынаюць з паўфіналу. 
        На гэтай стадыі трэцяе месца гуляе з шостым, чацвёртае "--- з пятым.
        \item Паўфінал: матч да 4 перамог. Пераможца пары 3-6 гуляе з другім месцам, пераможца 4-5 "--- з першым.
        \item Фінал: серыя да двух перамог, кожны матч гуляецца да трох выйграных партый.
    \end{itemize}

\end{document}